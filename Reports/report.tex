\documentclass{article}

\usepackage{graphicx}

\title{SSI WP3: Project report}

\author{Tim De Jong}

\begin{document}

\maketitle


\section{Introduction}

Process:
- Data preparation
- Data analysis
- Train ML classifier
- Evaluate ML classifier:
    - After training
    - On new data (for instance different supermarket)
    - Over time (train on one year, evaluate on next year)




Train a ML classifier to classify receipt texts to COICOP labels:
- Several levels of COICOP
- Receipt text for Plus and LIDL supermarkets
- Encode receipt text as a vector, several methods: 
    - Several encodings have different qualities, evaluate quality of these encodings
    - Which encoding/embedding works the best?
    - How do vector distances differ between COICOP categories? How? Which base-model?
- Do receipt text change COICOP label and if so how often?
- How often do unique identifiers such as EAN codes change? On what COICOP level do they change?
    
Can we find new products easily? Outlier/Novelty detection methods:    
    - Cluster on vectors of receipt text

- Train on one supermarket, evaluate on other supermarket
- Which product categories are most difficult to classify?
- Which product caterogies are classified best/worst?
- How do supermarkets differ in their product categories?

String distances:
- How do string distances differ between COICOP categories?

Manual labeling by participants:
- How do we show participants a list of relevant COICOP categories?
- Which level of COICOP category?
- Can we evaluate the quality of the manual labels? By comparing random selection of top k-labels with a 
bootstrap method?

From previous work it became clear that ML classifier performance degrades over time
- Evaluate degradation of performance over time:
    - Train on one year, evaluate on next year




\end{document}